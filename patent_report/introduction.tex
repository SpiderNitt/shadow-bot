\section{Introduction}

\IEEEPARstart{A}{TR}s  belong to the category of mobile robots which are mainly used for off-road purposes. They usually have wheels or tracks for locomotion. But existing designs have a limitation on the obstacle they can avoid, ie they cannot traverse in a place with really high almost vertical surfaces on all sides. Quite a lot of research has been carried out in the previous decades in this field and excellent ATRs and wall climbing bots have been developed separately.

In the existing Wall climbing bots, an external help is need in order to initially make them fixed to the wall. Several approaches have developed to make the bots move on the walls. Primitive designs use pneumatic suction caps at the end of the actuators to make them fixed on the walls. If the surface is made of ferro-magnetic materials then an electromagnetic actuator is used which gets attached and detached from the surface as the magnet is energized and de-energized. One of the recent bots developed by MIT uses \textit{bio-mimicry} inspired by \textit{Geckos} which have hair like protrusions in order to better grip the surface. But this surface alterations usually involves a cost as nano-materials have to embedded on them. Thus existing bots have a difficulty in commercialization and developed into a product.

Conventional suspensions involve a lot of mechanical components like springs, dampers and struts. The problem with such a system is that a spring has a fixed spring constant and damping constant value arising due to physical structure and material of the spring components. Thus the dynamics of such a system cannot be varied except that a threshold can be set on the maximum load on the spring. Such a spring system cannot be used in an ATR as it encounters new situations depending on the environment it is subjected to. Variable stiffness suspensions are available in the market, but they are expensive and also provide a range of working loads. As loads increase the cost also increase exponentially.

Until recently no bot have been developed with the ability to perform a ground to wall transition autonomously  except \textit{Vertigo}, a bot jointly designed by researchers at \textit{ETH Zurich} and \textit{Disney Research}. Vertigo uses 2 tiltable propellers to maneuver itself in the ground and walls. It has a full carbon fibre frame and servo actuators to rotate the propellers. The carbon fibre frame makes it costly and it also suffers from \textit{gyroscopic precession} which gives a slight jerk whenever the bot tries to change its direction.

The goal of this project is to design an ATR which can be used in any kind of rough terrain with variable size obstacle negotiation capability and dynamic environment adaptability without comprising on the stability at all times. The bot design has been discussed in detail in section II.The Dynamic Analysis, Wall climbing and Adaptive Suspension are discussed in Sections III, IV and V respectively. 
